\anonsection{Вывод}

В ходе выполнения индивидуального задания была проведена двухсторонняя интеграция языков программирования Lua и C++ с использованием IDE VSCodium и кроссплатформенного инструмента сборки CMake. Разработан программный код, который демонстрирует успешную работу интеграции. В процессе работы над проектом были выполнены следующие задачи:

\begin{enumerate}
	\item \textbf{Изучение и применение принципов интеграции Lua с C++}. Были изучены основные подходы к взаимодействию между Lua и C++ с использованием библиотеки Lua. Созданы примеры кода, демонстрирующие вызов Lua-скриптов из C++ и управление C++ объектами в Lua.

	\item \textbf{Настройка рабочей среды и инструментов сборки}. Для разработки был использован редактор VSCodium, который обеспечил удобное и эффективное написание кода. Были настроены инструменты сборки CMake, что позволило автоматизировать процесс компиляции и управления зависимостями между библиотеками Lua и C++.

	\item \textbf{Создание и тестирование интеграции Lua и C++}. Реализована двухсторонняя интеграция, при которой C++ может вызывать Lua-функции, а Lua может работать с C++ объектами. Это дало возможность реализовать гибкое взаимодействие между языками программирования, где C++ используется для высокой производительности и работы с системными ресурсами, а Lua -- для написания высокоуровневой логики работы приложения.

	\item \textbf{Защита и обоснование применения}. Для защиты отчета был подготовлен доклад, обосновывающий необходимость двухсторонней интеграции Lua и C++ в разработке программного обеспечения, а также ее применение в реальных задачах, требующих гибкости в программировании и высокой производительности.
\end{enumerate}

Результаты выполнения задания продемонстрировали успешное использование двух языков программирования для решения задач, требующих как гибкости скриптов, так и производительности системного программирования.

\clearpage