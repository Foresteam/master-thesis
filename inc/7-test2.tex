\section{Интеграция C++ в Lua}

\subsection{Сборка}
Помимо использования Lua из C++, что является более целесообразным, в виду скудности инструментария встроенных библиотек Lua, существует также возможность обратной интеграции. В этом случае, исходный код C++ компилируется в библиотеку (статической либо динамической компоновки), после чего подключается и используется из Lua.

Для осуществления этой обратной интеграции, структуру проекта нужно изменить следующим образом:
\addmyimage{images/2-1.png}{2-folder}{Файловая структура проекта обратной интеграции}

Основные изменения:
\begin{itemize}
	\item Функции таймеров и поиска простых чисел вынесены в отдельные файлы.
	\item CMakeLists.txt был переписан для сборки проекта в динамическую библиотеку (.so).
	\item Проект будет собираться в библиотеку динамической компоновки, .so, а не в исполняемый файл.
	\item Вместо LuaFunctions теперь есть только main.cpp, который и управляет регистрацией функций библиотеки.
\end{itemize}

Основные изменения в \autoref{lst:CMakeListsR.txt}:
\begin{itemize}
	\item Строка с add\_executable была изменена на add\_library, был добавлен флаг SHARED, чтобы проект собирался в файл .so.
\end{itemize}

Также, чтобы проект собирался в .so, нужно было внести изменения в файл проекта библиотеки Lua, lua/lua-5.4.4/CMakeLists.txt, и добавить следующий флаг сборки:
\addmyimage{images/2-2.png}{2-flag}{Флаг для сборки проекта обратной интеграции в .so}

Сама сборка проекта производится так же, как и при интеграции Lua в C++:
\addmyimage{images/2-3.png}{2-builf}{Сборка проекта обратной интеграции}
\clearpage

\subsection{Тестирование}
Тестовый код будет делать то же самое, что и при обычной интеграции: вызывать функцию нахождения простых чисел и засекать время выполнения:
\addmyimage{images/2-5.png}{}{Код скрипта Lua}

Теперь исполняемой средой будет непосредственно runtime Lua, предварительно установленный в системе. Запуск происходит следующей командой:
\begin{lstlisting}
	lua script.lua
\end{lstlisting}

В результате, получается следующий вывод в консоли:
\addmyimage{images/2-4.png}{2-result}{Результат запуска собственной библиотеки для Lua}

Как видим, простые числа были успешно найдены, за то же время, что и в главе 6.
\clearpage