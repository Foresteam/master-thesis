\subsection{Проектирование приложения}

\subsubsection{Средства разработки}
Для разработки данного проекта были выбраны следующие средства:

\begin{enumerate}
	\item VSCodium -- IDE. Является OpenSource версией проприетарного редактора кода VSCode, позиционирующего себя, как легковесную среду разработки. Поддерживает множество языков  и инструментов программирования, в том числе C++ и CMake, с выбором компилятора (CLang).
	\item Соответсвующие расширения для VSCodium: для C++, CMake, Lua и JavaScript.
	\item CMake -- кроссплатформенный инструмент для создания приложений, поддерживающий различные компиляторы. Связующее звено проекта.
	\item CLang -- компилятор языка C++. На данный момент более перспективен, чем GCC.
	\item lua 5.4.4 -- библиотека Lua для C++ \cite{lua_cmake}.
	\item plusaes -- библиотека для работы с AES в C++.
	\item foresteamnd -- собственная библиотека, предоставляющая ряд утилитарных функций и упрощенных реализаций некоторых механизмов \cite{foresteamnd}.
\end{enumerate}

Стоит отдельно упомянуть, что оригинальный Lua написан на C, собирается через обычный makefile, да еще и в библиотеку динамической компоновки (.so). Это не только затруднило бы работу с ним, но и перечеркнуло портативность: пришлось бы поставлять библиотеку Lua вместе с приложением, либо рассчитывать, что у пользователя она уже установлена. Поэтому вместо оригинального Lua была взята библиотека, уже переписанная под сборку через CMake \cite{lua_cmake}, которая, к тому же, умеет собираться статически.
\clearpage

\subsubsection{Дерево файлов приложения}
Директория проекта приложения имеет следующую структуру:
\begin{itemize}
	\item include -- Подключаемые файлы библиотек
	\item lib -- Собранные файлы библиотек
	\item lua -- Проект Lua 5.4.4 на CMake
	\item plusaes -- Репозиторий библиотеки plusaes
	\item CMakeLists.txt -- Основной файл проекта CMake
	\item src -- Исходный код проекта
	\begin{itemize}
		\item lua -- Исходный код скриптов lua, конвертированные в .h файлы этих скриптов, ключ AES, вспомогательный инструмент для конвертации скриптов в зашифрованные файлы C++.
		\item main.cpp -- Основной файл приложения.
		\item LuaFunctions.h -- Заголовочный файл экспортируемых в Lua функций.
		\item LuaFunctions.cpp -- Файл, отвечающий за экспорт функций в Lua.
	\end{itemize}
	\item build -- Продукты сборки проекта.
\end{itemize}

\subsubsection{Механизм сборки приложения}
За сборку проекта отвечает файл CMakeLists.txt (\autoref{lst:CMakeLists.txt}). Данный файл представляет собой скрипт CMake. Этот CMake-скрипт организует сборку проекта, который использует C++ и Lua. Он включает компиляцию файлов Lua с помощью Node.js, создает исполняемый файл rut, который использует скомпилированные Lua-файлы, и настраивает различные платформы и зависимости для успешной сборки. Шаги сборки:
\begin{enumerate}
	\item Название проекта: rut, v0.1.0.
	\item Добавление подпроекта: lua
	\item Добавление пользовательского продукта сборки -- файлов C++, конвертированных из скриптов Lua, а также файла ключа.
	\item Листинг файлов проекта, в порядке их компиляции и сборки.
	\item Добавление продуктов сборки скриптов Lua как зависимостей проекта.
	\item Добавление директорий включения (include).
	\item Компоновка библиотек статических библиотек: foresteamnd и lua\_static.
\end{enumerate}
На выходе получается файл \textit{rut}, и ряд побочных продуктов сборки, вроде makefile в директории build проекта.

\clearpage