\anonsection{Введение}

Производственная практика (проектная) направлена на приобретение навыков и опыта самостоятельного решения практических задач, закрепление и углубление теоретической подготовки обучающегося, а также развитие практических навыков и компетенций, необходимых для работы с современными языками программирования и инструментами разработки.

Актуальность данной работы обусловлена растущей потребностью в эффективной интеграции различных языков программирования в условиях стремительного развития технологий. Одной из таких актуальных задач является интеграция языков C++ и Lua, что позволяет создавать мощные и гибкие решения для различных областей применения. Возможность взаимодействия C++ и Lua предоставляет разработчикам инструменты для расширения функциональности приложений, их оптимизации и упрощения разработки за счет использования скриптовых возможностей Lua.

Целью данной практики является изучение механизмов интеграции языков программирования C++ и Lua в среде операционной системы Linux, а также разработка инструмента, который будет осуществлять конвертацию Lua-скриптов в файлы исходного кода C++.

Задачами производственной (проектной) практики являются:
\begin{itemize}
	\item изучение литературы по интеграции C++ и Lua, а также технологий, используемых для этой цели;
	\item разработка вспомогательного инструмента для конвертации Lua-скриптов в C++-код;
	\item выполнение двухсторонней интеграции Lua и C++ с использованием IDE VSCodium и системы сборки CMake;
	\item разработка программного кода, демонстрирующего успешную интеграцию и взаимодействие двух языков.
\end{itemize}

Данная работа способствует закреплению теоретических знаний и навыков проектирования информационных процессов, а также приобретению практического опыта в области программирования и интеграции современных технологий.

\clearpage
