\section{Изучение подходов к проектной работе в IT-индустрии}
Методологии разработки ПО представляют собой совокупность
определенных подходов, принципов и инструментов, которые позволяют
управлять проектами разработки для достижения поставленных целей \cite{metod-razr}.
Основные цели внедрения методологий разработки ПО:
\begin{itemize}
	\item увеличение эффективности ПО;
	\item улучшение качества ПО;
	\item достижение высокого уровня прозрачности и управляемости проекта;
	\item улучшение коммуникации и сотрудничества;
	\item достижение гибкости и адаптивности.
\end{itemize}
Ниже перечислены основные методологии разработки ПО:

\begin{enumerate}
	\item \textbf{Waterfall Model} (каскадная модель или «водопад»). Каскадная модель разработки программного продукта — это классический подход к разработке ПО, который предполагает линейное выполнение последовательных этапов разработки: сбор требований, проектирование, реализацию, тестирование и сопровождение. Каждый этап начинается только после успешного завершения предыдущего, то есть в этой модели нет возможности вернуться на предыдущие этапы.
	В каскадной модели каждый этап разработки имеет свою документацию и четкие требования, которые должны быть выполнены, прежде чем можно перейти к следующему этапу. Это делает каскадную модель более предсказуемой и позволяет лучше контролировать процесс разработки, особенно в случаях, когда требования к продукту уже четко определены.
	Однако, такой линейный подход также имеет свои недостатки, например, сложность изменения требований на поздних стадиях проекта или необходимость повторения предыдущих этапов при обнаружении ошибок. Кроме того, каскадная модель не учитывает возможности быстро реагировать
	на изменения, которые могут произойти в процессе разработки.
	
	\item \textbf{V-модель} разработки программного продукта — это модель, которая является улучшенной версией каскадной модели. В V-модели процесс разработки представлен в виде буквы «V», где каждый этап проекта имеет соответствующий этап тестирования. Основная идея V-модели заключается в том, что качество продукта зависит от качества его тестирования. Каждый этап разработки имеет соответствующий этап тестирования, начиная с тестирования требований и заканчивая тестированием внедрения. Преимуществом V-модели является то, что она помогает обнаружить ошибки и дефекты на ранних этапах разработки, что может существенно сократить время и затраты на доработку продукта. Это также делает процесс разработки более предсказуемым и улучшает контроль качества. Однако, также как и каскадная модель, V-модель не предусматривает возможности быстро адаптироваться к изменениям в требованиях к продукту, что может ограничивать гибкость разработки в динамичных средах. В таких случаях лучше использовать более гибкие методологии, такие как Agile или Scrum.
	
	\item \textbf{Agile} – это философия разработки программного обеспечения, которая призывает к гибкости и адаптивности в процессе разработки продукта. Она акцентирует внимание на быстрой итеративной разработке продукта с частыми поставками рабочего программного кода, которые постоянно улучшаются на основе обратной связи от пользователей и заказчиков. Методологии Agile, такие как Scrum, Kanban, Extreme Programming (XP), Feature Driven Development (FDD) и другие, позволяют командам быстро реагировать на изменения требований и перестраивать свой подход к разработке, чтобы достичь лучших результатов. В центре философии Agile лежит идея о том, что процесс разработки должен быть гибким и адаптивным, что позволяет командам быстро реагировать на изменения и принимать важные решения на основе обратной связи от пользователей и заказчиков. В Agile-разработке большое внимание уделяется коммуникации внутри команды и с заказчиками, автоматизации тестирования, постоянной интеграции и обновлению кода, улучшению процессов и инструментов разработки, а также нахождению наилучших практик и оптимизации всего процесса разработки. Благодаря этому подходу Agile-разработка позволяет снизить время и стоимость разработки, увеличить качество и улучшить удовлетворенность пользователей.
	
	\item \textbf{RAD} (Rapid Application Development) — разновидность инкрементной модели, которая акцентирует внимание на быстрой разработке продукта в условиях сильных ограничений по срокам и бюджету и нечётко определённых требований к продукту. Эффект ускорения разработки достигается путём непрерывного, параллельного с ходом разработки, уточнения требований и оценки текущих результатов с привлечением заказчика.
	
	\item \textbf{Модель Spiral} (спиральная модель) — это гибкая методология разработки программного обеспечения, которая сочетает в себе итеративный подход с последовательностью шагов, основанных на рисках. Основная идея спиральной модели заключается в том, чтобы разбить проект на более мелкие итерации, каждая из которых содержит этапы планирования, анализа рисков, проектирования, реализации, тестирования и оценки. Каждая итерация зависит от предыдущей и планируется на основе ее результатов. Спиральная модель включает в себя оценку рисков на каждой стадии проекта, а также их управление. Это помогает снизить риски и улучшить качество продукта, который создается в процессе разработки. В данном проекте используется методология разработки Scrum (\autoref{scrum-schema}), основанная на принципах Agile \cite{best-metod-razr}. Проект разделяется на небольшие задачи, которые необходимо сделать за определенное количество времени (эти временные интервалы, как и в Agile, называются спринтами). Модель разработки ПО Scrum построена таким образом, чтобы помочь командам естественным образом адаптироваться к меняющимся условиям рынка и потребностям пользователей. В то же время короткие циклы позволяют разработчикам быть эффективнее. Scrum обеспечивает структуру, оптимизирует разработку и при этом остается гибким и учитывает желания владельца продукта.
\end{enumerate}


\addmyimage{images/scrum.png}{scrum-schema}{Схема методологии Scrum}

Разработка программного обеспечения на C++ в операционной системе Linux с использованием CMake — это современный подход к созданию кроссплатформенных приложений, который позволяет упрощать управление проектом и процесс сборки. CMake является мощным инструментом для автоматизации сборки, настройки параметров компиляции, интеграции библиотек и организации структуры кода \cite{cmake-habr}\cite{cmake-docs}.

CMake работает как генератор файлов сборки, создавая конфигурации для таких инструментов, как Make, Ninja или Visual Studio. Это делает его идеальным выбором для проектов на C++, которые требуют поддержки разных платформ и конфигураций \cite{cmake-docs}.
Основные этапы разработки на C++ с использованием CMake:

\begin{enumerate}
	\item \textbf{Инициализация проекта}. Проект начинается с создания файла CMakeLists.txt, который описывает настройки сборки, исходные файлы и зависимости. Этот файл указывает минимальную версию CMake, имя проекта и исполняемый файл, который будет создан \cite{cmake-docs}.
	
	\item \textbf{Создание сборочной директории}. Принято разделять исходный код и файлы сборки, используя отдельную папку. Этот подход упрощает управление проектом, так как все временные файлы находятся в одном месте \cite{cmake-habr}.
	
	\item \textbf{Добавление библиотек}. Одним из ключевых преимуществ CMake является простое подключение внешних библиотек. Это делает проект гибким и позволяет легко интегрировать дополнительные модули \cite{cmake-docs}.
	
	\item \textbf{Тестирование и настройка}. CMake поддерживает автоматизированное тестирование, что особенно важно для больших проектов. Инструменты вроде CTest упрощают создание тестов и интеграцию их в процесс сборки \cite{cmake-docs}.
	
	\item \textbf{Оптимизация}. Для улучшения производительности на этапе компиляции можно включить оптимизационные флаги.
\end{enumerate}

\textbf{Преимущества использования CMake}:
\begin{itemize}
	\item Кроссплатформенность: CMake поддерживает Windows, Linux и macOS, что позволяет разрабатывать универсальные приложения \cite{cmake-habr}.
	\item Модульность: Удобная интеграция внешних библиотек и модулей.
	\item Автоматизация: Упрощение процессов сборки и тестирования \cite{cmake-docs}.
\end{itemize}

\clearpage
