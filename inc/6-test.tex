\subsection{Тестирование работы приложения}

\subsubsection{Сборка приложения}
Сборка приложения выполняется в 2 этапа:
\begin{enumerate}
	\item Переход в директорию сборки и создание файлов сборки (\autoref{build-prepare}).
	\item Сборка приложения (\autoref{build-build}).
\end{enumerate}

\addmyimage{images/1.png}{build-prepare}{Подготовка файлов сборки приложения}
\addmyimage{images/2.png}{build-build}{Сборка приложения}
\clearpage

\subsubsection{Запуск приложения по сценарию 1}
Для подготовки потребовалось:
\begin{enumerate}
	\item Сделать идентичные функции поиска простых чисел в C++ и Lua.
\end{enumerate}
\addmyimage{images/1-1-1.png}{}{Функция поиска простых чисел в C++}
\addmyimage{images/1-1-2.png}{}{Функция поиска простых чисел в Lua}
\addmyimage{images/1-1-3.png}{}{Функция запуска функции простых чисел из Lua в C++}

\begin{enumerate}[resume]
	\item Сделать в Lua функцию проверки операционной системы (Windows/Unix).
\end{enumerate}
\addmyimage{images/1-1-4.png}{}{Функция проверки операционной системы на принадлежность к семейству Unix}

\begin{enumerate}[resume]
	\item Сделать в C++ команду для запуска программы из оболочки (bash), а также 2 команды для работы с таймером: его запуск с возвращением ключа для дальнейшего обращения, и его завершение, которое возвращает время, прошедшее с запуска. Добавить команду для показа системного окна уведомлений.
\end{enumerate}

Далее, для тестирования работы, было произведено следующее:
\begin{enumerate}
	\item Вызвана системная команда <<neofetch>>, отображающая сведения о системе в вывод.
	\item Показано системное окно с текстом (\autoref{alert-test}).
	\item Выведено, на каком семействе операционных систем запущена программа.
	\item Вызвана C++ функция поиска простых чисел, выведен результат и затраченное время.
	\item Далее, из C++ вызвана Lua-функция IsWindows, и так же выведен результат о семействе операционной системы.
	\item Запущена из C++ Lua-функция поиска простых чисел, выведен ее результат и затраченное время.
\end{enumerate}

Финальный результат работы: \autoref{final-result}.

\addmyimage{images/3.png}{alert-test}{Результат работы программы 1. Системное информационное окно}
\addmyimage{images/4.png}{final-result}{Результат работы программы 1. Вывод в консоль}
\clearpage

\subsubsection{Запуск приложения по сценарию 2}
Запуск скрипта из скрипта.

Мы можем также использовать среду Lua, чтобы динамически запускать из нее сценарии. Для этого, добавим следующие функции в C++ и в Lua:
\addmyimage{images/1-2-1.png}{}{Функция выполнения скрипта в C++}
\addmyimage{images/1-2-2.png}{}{Функция выполнения скрипта в Lua}

Теперь, получим сценарий от пользователя из консоли:
\addmyimage{images/1-2-3.png}{}{Выполнение пользовательского сценарий в окружении Lua}

Запустим программу:
\addmyimage{images/1-2-4-1.png}{}{Результат работы программы 2. Системное информационное окно}
\addmyimage{images/1-2-4-2.png}{}{Результат работы программы 2. Вывод в консоль}


\clearpage