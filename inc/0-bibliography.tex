\newcommand{\bibitemsite}[4]{
	\bibitem{#1} #2 -- [Электронный ресурс] URL: \url{#3} (дата обращения: #4)
}

\begingroup
\renewcommand{\section}[2]{\anonsection{Библиографический список}}
\begin{thebibliography}{00}

\bibitemsite{metod-razr}
	{Методологии разработки программного продукта / Testengineer}
	{https://testengineer.ru/development-methodologies-sys-anal/}
	{18.12.2024}

\bibitemsite{best-metod-razr}
	{8 лучших методологий разработки ПО в 2024 году / Purrweb}
	{https://www.purrweb.com/ru/blog/metodologii-dlya-razrabotki-po/}
	{18.12.2024}

\bibitemsite{cmake-habr}
	{CMake и C++ — братья навек / Хабр}
	{https://habr.com/ru/articles/461817/}
	{18.12.2024}

\bibitemsite{cmake-docs}
	{CMake documentation and community}
	{https://cmake.org/documentation/}
	{18.12.2024}

\bibitemsite{lua_c1}
  {Что такое скрипты и с чем их едят}
  {https://habr.com/ru/articles/196272/}
  {18.12.2024}

\bibitemsite{lua_c2}
  {LuaBridge 3.0 Reference Manual}
  {https://kunitoki.github.io/LuaBridge3/Manual}
  {18.12.2024}

\bibitemsite{lua_c3}
  {Integrating Lua in C++ - GeekForGeeks}
  {https://www.geeksforgeeks.org/integrating-lua-in-cpp/}
  {18.12.2024}

\bibitemsite{foresteamnd}
	{foresteamnd - Super useful library for C++ - GitHub}
	{https://github.com/Foresteam/foresteamnd}
	{18.12.2024}

\bibitemsite{lua_cmake}
  {CMake-based build of Lua (5.4.6 and 5.3.3) - GitHub}
  {https://github.com/walterschell/Lua}
  {18.12.2024}

\end{thebibliography}
\endgroup

\clearpage
