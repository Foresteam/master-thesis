\section{Интеграция Lua в C++}

\subsection{Разработка и создание вспомогательного инструмента, осуществляющего конвертацию скрипта Lua в файл исходного кода C++}
\subsubsection{Обоснование необходимости}
Данный шаг, строго говоря, не является обязательным. Однако дает ряд неоспоримых преимуществ:
\begin{itemize}
	\item Компактность. Программу можно будет поставлять в виде одного файла (особенно при условии статической компоновки).
	\item Гибкость и удобство. К работе будет предоставлен не файл, который надо читать с диска, проверять наличие, ошибки и т.п, а строка в готовом и родном для C виде (массив символов). В таком случае, ошибки будут отсеиваться на этапе компиляции.
	\item Безопасность. Файл скрипта можно будет зашифровать, сделав код всей программы закрытым. За счет этого, он будет так же труден для изменения и чтения, как и обычная скомпилированная программа на C.
\end{itemize}

\subsubsection{Разработка и создание вспомогательного инструмента}
В качестве средств разработки были выбраны:
\begin{itemize}
	\item JS -- скриптовый язык программирования.
	\item VSCodium -- IDE.
	\item NodeJS -- интерпретатор языка JavaScript.
	\item js-crypto-aes -- библиотека для работы с криптографией.
\end{itemize}
Сам проект представляет собой проект на NodeJS (\autoref{lst:package.json}). Состоит из двух исполняемых файлов:
\begin{itemize}
	\item key.js (\autoref{lst:key.js}) -- файл с ключом AES.
	\item compile.js (\autoref{lst:compile.js}) -- основной файл, осуществляющий преобразование скриптов Lua в удобный для работы в C++ формат; шифрование исходного Lua-кода, для его последующей расшифровки уже во время исполнения приложения на C++.
\end{itemize}
Помимо преобразования Lua файлов, проект также осуществляет конвертацию ключа в формат, удобный для работы в C++.

Алгоритм работы программы:
\begin{enumerate}
	\item Чтение файлов Lua: Она ищет все файлы с расширением .lua в текущей директории.
	
	\item Обработка содержимого Lua файлов:
	
	\begin{itemize}
		\item Читает каждый Lua файл и преобразует его содержимое в строку C++ с добавлением экранированных символов для новой строки (\texttt{\textbackslash n}).
		\item Удаляет пустые строки и строки, начинающиеся с комментариев.
		\item Преобразует фигурные скобки и запятые в формат, совместимый с C++.
	\end{itemize}
	
	\item Шифрование содержимого:
	
	\begin{itemize}
		\item Преобразует строку в байты и шифрует с использованием AES (режим CBC) с заранее сгенерированным ключом и вектором инициализации (IV).
	\end{itemize}
	
	\item Запись зашифрованного скрипта в файл C++:
	
	\begin{itemize}
		\item Записывает зашифрованный скрипт в C++ заголовочный файл в виде строки (с кодировкой байтов как).
		\item Добавляет информацию о длине зашифрованного содержимого.
	\end{itemize}
	
	\item Запись AES ключа в файл C++:
	
	\begin{itemize}
		\item Записывает сгенерированные значения для salt, key, и iv в отдельный C++ заголовочный файл.
	\end{itemize}
\end{enumerate}
\clearpage