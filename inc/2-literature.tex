\section{Изучение литературы по интеграции ЯП C++ и Lua в ОС Linux}
Интеграция языков программирования C++ и Lua является важным вопросом, потому что: C++ является одним из самых популярных языков для создания высокопроизводительных приложений, в то время как Lua представляет собой легковесный, высокоэффективный скриптовый язык, который активно используется для расширения функциональности и настройки программ.

В рамках интеграции Lua в C++ приложения особое внимание уделяется таким библиотекам, как LuaBridge, которая упрощает взаимодействие между этими языками. LuaBridge 3.0 предоставляет обширную документацию и примеры использования, что делает процесс интеграции удобным и доступным для разработчиков \cite{lua_c2}. Данная библиотека не является единственной в своем роде, и имеет аналоги \cite{lua_c3}.

В статье <<Что такое скрипты и с чем их едят>> на платформе Habr, объясняется принцип работы с Lua в контексте разработки скриптов, а также описываются основные возможности и области применения скриптовых языков в C++ проектах, в том числе в операционных системах Linux. Lua предлагает простоту в использовании и гибкость, что делает его популярным выбором для внедрения в C++ приложения для динамической настройки и расширения функционала \cite{lua_c1}.

Причем в приведенной статье рассматривается именно работа с <<голым>> Lua, что предпочтительно.
\clearpage