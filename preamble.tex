%%% Преамбула %%%

\usepackage{fontspec} % XeTeX
\usepackage{xunicode} % Unicode для XeTeX
\usepackage{xltxtra}  % Верхние и нижние индексы
\usepackage{pdfpages} % Вставка PDF

\usepackage{xcolor}
\usepackage{listings} % Оформление исходного кода
\lstset{
basicstyle=\small\ttfamily, % Размер и тип шрифта
breaklines=true,            % Перенос строк
tabsize=2,                  % Размер табуляции
frame=single,               % Рамка
literate={--}{{-{}-}}2,     % Корректно отображать двойной дефис
literate={---}{{-{}-{}-}}3,  % Корректно отображать тройной дефис
literate=
	{а}{{\cyra}}1 {б}{{\cyrb}}1 {в}{{\cyrv}}1 {г}{{\cyrg}}1 {д}{{\cyrd}}1
{е}{{\cyre}}1 {ё}{{\cyryo}}1 {ж}{{\cyrzh}}1 {з}{{\cyrz}}1 {и}{{\cyri}}1
{й}{{\cyrishrt}}1 {к}{{\cyrk}}1 {л}{{\cyrl}}1 {м}{{\cyrm}}1 {н}{{\cyrn}}1
{о}{{\cyro}}1 {п}{{\cyrp}}1 {р}{{\cyrr}}1 {с}{{\cyrs}}1 {т}{{\cyrt}}1
{у}{{\cyru}}1 {ф}{{\cyrf}}1 {х}{{\cyrh}}1 {ц}{{\cyrc}}1 {ч}{{\cyrch}}1
{ш}{{\cyrsh}}1 {щ}{{\cyrshch}}1 {ъ}{{\cyrhrdsn}}1 {ы}{{\cyrery}}1
{ь}{{\cyrsftsn}}1 {э}{{\cyrerev}}1 {ю}{{\cyryu}}1 {я}{{\cyrya}}1
{А}{{\CYRA}}1 {Б}{{\CYRB}}1 {В}{{\CYRV}}1 {Г}{{\CYRG}}1 {Д}{{\CYRD}}1
{Е}{{\CYRE}}1 {Ё}{{\CYRYO}}1 {Ж}{{\CYRZH}}1 {З}{{\CYRZ}}1 {И}{{\CYRI}}1
{Й}{{\CYRISHRT}}1 {К}{{\CYRK}}1 {Л}{{\CYRL}}1 {М}{{\CYRM}}1 {Н}{{\CYRN}}1
{О}{{\CYRO}}1 {П}{{\CYRP}}1 {Р}{{\CYRR}}1 {С}{{\CYRS}}1 {Т}{{\CYRT}}1
{У}{{\CYRU}}1 {Ф}{{\CYRF}}1 {Х}{{\CYRH}}1 {Ц}{{\CYRC}}1 {Ч}{{\CYRCH}}1
{Ш}{{\CYRSH}}1 {Щ}{{\CYRSHCH}}1 {Ъ}{{\CYRHRDSN}}1 {Ы}{{\CYRERY}}1
{Ь}{{\CYRSFTSN}}1 {Э}{{\CYREREV}}1 {Ю}{{\CYRYU}}1 {Я}{{\CYRYA}}1
}
\definecolor{darkgreen}{rgb}{0,0.5,0} % Custom dark green color
\lstdefinelanguage{TypeScript}{
	keywords={typeof, new, true, false, catch, function, return, null, catch, switch, var, if, in, for, of, while, do, else, case, break, const, let, import, export, default, class, extends, async, await, from, interface },
	ndkeywords={class, export, boolean, number, string, object, throw, implements, import, this, from, interface },
	sensitive=false,
	comment=[l]{//},
	morecomment=[s]{/*}{*/},
	keywordstyle=\color{blue}\bfseries,
	ndkeywordstyle=\color{darkgray}\bfseries,
	commentstyle=\color{darkgreen}\ttfamily,
	stringstyle=\color{red}\ttfamily,
	morestring=[b]',
	morestring=[b]"
}
% Define Vue.js language
\lstdefinelanguage{Vue}{
	keywords={import, export, default, from, interface },
	morekeywords={template,script,data,methods,computed,props,watch,mounted,created,updated,destroyed,setup,ref,reactive},
	ndkeywords={ export, boolean, import, this, from, interface },
	sensitive=true,
	morecomment=[l]{//},
	morecomment=[s]{/*}{*/},
	morecomment=[s]{<!--}{-->},
	morestring=[b]",
	morestring=[b]',
	morestring=[b]`,
	keywordstyle=\color{blue}\bfseries,
	ndkeywordstyle=\color{darkgray}\bfseries,
	commentstyle=\color{darkgreen}\ttfamily,
	stringstyle=\color{red}\ttfamily,
	tabsize=2,
	frame=single,
	showstringspaces=false,
	breaklines=true
}

% Шрифты, xelatex
\defaultfontfeatures{Ligatures=TeX}
\setmainfont{Times New Roman} % Нормоконтроллеры хотят именно его
\newfontfamily\cyrillicfont{Times New Roman}
% \setsansfont{Liberation Sans} % Тут я его не использую, но если пригодится
\setmonofont{FreeMono} % Моноширинный шрифт для оформления кода

% Формулы
\usepackage{mathtools,unicode-math} % Не совместим с amsmath
\setmathfont{XITS Math}             % Шрифт для формул: https://github.com/khaledhosny/xits-math
\numberwithin{equation}{section}    % Формула вида секция.номер

% Русский язык
\usepackage{polyglossia}
\setdefaultlanguage{russian}

% Абзацы и списки
\usepackage{enumerate}   % Тонкая настройка списков
\usepackage{indentfirst} % Красная строка после заголовка
\usepackage{float}       % Расширенное управление плавающими объектами
\usepackage{multirow}    % Сложные таблицы

% Пути к каталогам с изображениями
\usepackage{graphicx} % Вставка картинок и дополнений
\graphicspath{{images/}}

% Формат подрисуночных записей
\usepackage{chngcntr}

% Сбрасываем счетчик таблиц и рисунков в каждой новой главе
\counterwithin{figure}{section}
\counterwithin{table}{section}
\AtBeginDocument{\counterwithin{lstlisting}{section}}

% Гиперссылки
\usepackage{hyperref}
\hypersetup{
	colorlinks, urlcolor={black}, % Все ссылки черного цвета, кликабельные
	linkcolor={black}, citecolor={black}, filecolor={black},
	pdfauthor={Амет Умеров},
	pdftitle={Исследование процессов обеспечения безопасности облачных сред}
}

% Оформление библиографии и подрисуночных записей через точку
\makeatletter
\renewcommand*{\@biblabel}[1]{\hfill#1.}
\renewcommand*\l@section{\@dottedtocline{1}{1em}{1em}}
\renewcommand{\thefigure}{\thesection.\arabic{figure}} % Формат рисунка секция.номер
\renewcommand{\thetable}{\thesection.\arabic{table}}   % Формат таблицы секция.номер
% \renewcommand{\thelstlisting}{\thesection.\arabic{lstlisting}} % Формат листинга: секция.номер
\def\redeflsection{\def\l@section{\@dottedtocline{1}{0em}{10em}}}
\makeatother

\renewcommand{\baselinestretch}{1.4} % Полуторный межстрочный интервал
\parindent 1.27cm % Абзацный отступ

\sloppy             % Избавляемся от переполнений
\hyphenpenalty=1000 % Частота переносов
\clubpenalty=10000  % Запрещаем разрыв страницы после первой строки абзаца
\widowpenalty=10000 % Запрещаем разрыв страницы после последней строки абзаца

% Отступы у страниц
\usepackage{geometry}
\geometry{left=3cm}
\geometry{right=1cm}
\geometry{top=2cm}
\geometry{bottom=2cm}

% Списки
\usepackage{enumitem}
% \setlist[enumerate,itemize]{leftmargin=\parindent} % Отступы в списках
\setlist[enumerate,itemize]{leftmargin=0pt, itemindent=2.5cm, labelsep=0.9cm} % Отступы в списках, как этого у нас хотят

\makeatletter
\AddEnumerateCounter{\arabic}{\@arabic}{м.}
\makeatother
\setlist{nolistsep}                           % Нет отступов между пунктами списка
\renewcommand{\labelitemi}{--}                % Маркер списка --
\renewcommand{\labelenumi}{\arabic{enumi}.}    % Список второго уровня
\renewcommand{\labelenumii}{\arabic{enumi}.\arabic{enumii}.} % Список третьего уровня

% Содержание
\usepackage{tocloft}
\renewcommand{\cfttoctitlefont}{\hspace{0.38\textwidth}\MakeTextUppercase} % СОДЕРЖАНИЕ
\renewcommand{\cftsecfont}{\hspace{0pt}}            % Имена секций в содержании не жирным шрифтом
\renewcommand\cftsecleader{\cftdotfill{\cftdotsep}} % Точки для секций в содержании
\renewcommand\cftsecpagefont{\mdseries}             % Номера страниц не жирные
\setcounter{tocdepth}{3}                            % Глубина оглавления, до subsubsection

% Список иллюстративного материала
\renewcommand{\cftloftitlefont}{\hspace{0.17\textwidth}\MakeTextUppercase}
\renewcommand{\cftfigfont}{Рисунок }
\addto\captionsrussian{\renewcommand\listfigurename{Список иллюстративного материала}}

% Список табличного материала
\renewcommand{\cftlottitlefont}{\hspace{0.2\textwidth}\MakeTextUppercase}
\renewcommand{\cfttabfont}{Таблица }
\addto\captionsrussian{\renewcommand\listtablename{Список табличного материала}}

% Нумерация страниц посередине сверху
\usepackage{fancyhdr}
\pagestyle{fancy}
\fancyhf{}
\cfoot{\textrm{\thepage}}
\fancyheadoffset{0mm}
\fancyfootoffset{0mm}
\setlength{\headheight}{17pt}
\renewcommand{\headrulewidth}{0pt}
\renewcommand{\footrulewidth}{0pt}
\fancypagestyle{plain}{
	\fancyhf{}
	\cfoot{\textrm{\thepage}}
}

% Формат подрисуночных надписей
\RequirePackage{caption}
\DeclareCaptionLabelSeparator{defffis}{ -- } % Разделитель
\captionsetup[figure]{justification=centering, labelsep=defffis, format=plain} % Подпись рисунка по центру
\captionsetup[table]{justification=raggedright, labelsep=defffis, format=plain, singlelinecheck=false} % Подпись таблицы слева
\captionsetup[lstlisting]{
	justification=centering,
	labelsep=defffis,
	format=plain
}
\addto\captionsrussian{
	\renewcommand{\figurename}{Рисунок}
	\renewcommand{\lstlistingname}{Листинг}
} % Имя фигуры

% Пользовательские функции
\newcommand{\addimg}[4]{ % Добавление одного рисунка
	\begin{figure}
		\centering
		\includegraphics[width=#2\linewidth]{#1}
		\caption{#3} \label{#4}
	\end{figure}
}
\newcommand{\addimghere}[4]{ % Добавить рисунок непосредственно в это место
	\begin{figure}[H]
		\centering
		\includegraphics[width=#2\linewidth]{#1}
		\caption{#3} \label{#4}
	\end{figure}
}
\newcommand{\addtwoimghere}[5]{ % Вставка двух рисунков
	\begin{figure}[H]
		\centering
		\includegraphics[width=#2\linewidth]{#1}
		\hfill
		\includegraphics[width=#3\linewidth]{#2}
		\caption{#4} \label{#5}
	\end{figure}
}

% Заголовки секций в оглавлении в верхнем регистре
\usepackage{textcase}
\makeatletter
\let\oldcontentsline\contentsline
\def\contentsline#1#2{
\expandafter\ifx\csname l@#1\endcsname\l@section
	\expandafter\@firstoftwo
\else
	\expandafter\@secondoftwo
\fi
{\oldcontentsline{#1}{\MakeTextUppercase{#2}}}
{\oldcontentsline{#1}{#2}}
}
\makeatother

% Оформление заголовков
\usepackage[compact,explicit]{titlesec}
\titleformat{\section}{}{}{12.5mm}{\centering{\thesection\quad\MakeTextUppercase{#1}}\vspace{1.5em}}
\titleformat{\subsection}[block]{\vspace{1em}}{}{12.5mm}{\thesubsection\quad#1\vspace{1em}}
\titleformat{\subsubsection}[block]{\vspace{1em}\normalsize}{}{12.5mm}{\thesubsubsection\quad#1\vspace{1em}}
\titleformat{\paragraph}[block]{\normalsize}{}{12.5mm}{\MakeTextUppercase{#1}}

% Секции без номеров (введение, заключение...), вместо section*{}
\newcommand{\anonsection}[1]{
	\phantomsection % Корректный переход по ссылкам в содержании
	\paragraph{\centerline{{#1}}\vspace{1em}}
	\addcontentsline{toc}{section}{#1}
}
% Секции без номеров (введение, заключение...), вместо section*{}
\newcommand{\anonsubsection}[1]{
	\phantomsection % Корректный переход по ссылкам в содержании
	\textnormal{#1}
	\addcontentsline{toc}{subsection}{#1}
}

% Секция для аннотации (она не включается в содержание)
\newcommand{\annotation}[1]{
	\paragraph{\centerline{{#1}}\vspace{1em}}
}

% Секция для списка иллюстративного материала
\newcommand{\lof}{
	\phantomsection
	\listoffigures
	\addcontentsline{toc}{section}{\listfigurename}
}

% Секция для списка табличного материала
\newcommand{\lot}{
	\phantomsection
	\listoftables
	\addcontentsline{toc}{section}{\listtablename}
}

% Создание новой нумерации для appsection (буквы A, B, C...)
\newcounter{appsection}[subsection] % Счетчик appsection зависит от subsection
\renewcommand{\theappsection}{\Asbuk{appsection}} % Буквенная нумерация

% Форматирование для appsection
\titleformat{\appsection}[block]{\vspace{1em}\normalsize}{}{12.5mm}{\theappsection\quad#1\vspace{1em}}

% Команда для appsection
\newcommand{\appsection}[1]{
	\refstepcounter{appsection} % Увеличиваем счетчик
	\phantomsection % Корректный переход по ссылкам в содержании
	\addcontentsline{toc}{subsection}{\theappsection\quad#1} % Добавляем в содержание
	\appsectionformat{#1} % Применяем форматирование
}

% Форматирование appsection
\newcommand{\appsectionformat}[1]{
	\vspace{1em}
	\normalsize
	\theappsection\quad#1
	\vspace{1em}
}

% Библиография: отступы и межстрочный интервал
\makeatletter
\renewenvironment{thebibliography}[1]
{\section*{\refname}
	\list{\@biblabel{\@arabic\c@enumiv}}
	{\settowidth\labelwidth{\@biblabel{#1}}
		\leftmargin\labelsep
		\itemindent 16.7mm
		\@openbib@code
		\usecounter{enumiv}
		\let\p@enumiv\@empty
		\renewcommand\theenumiv{\@arabic\c@enumiv}
	}
	\setlength{\itemsep}{0pt}
}
\makeatother

\usepackage{lastpage} % Подсчет количества страниц
\setcounter{page}{3}  % Начало нумерации страниц

% Названия ссылок рисунков
\addto\extrasrussian{
	\renewcommand{\figureautorefname}{Рисунок}
}
\renewcommand{\figureautorefname}{Рисунок}
% Названия ссылок листингов
\providecommand*{\lstlistingautorefname}{Листинг}

\newcommand{\addmyimage}[3]{
	\begin{figure}[H]
		\centering
		\includegraphics[width=1\textwidth]{#1}
		\caption{#3}
		\label{#2}
	\end{figure}
}